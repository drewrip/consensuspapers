\documentclass{article}
\begin{document}

\tableofcontents{}

\section{Verification and Checking}

\begin{itemize}
	%% Verdi
	\item
	\textit{Verdi: A Framework for Implementing and Formally Verifying Distributed Systems} \cite{Verdi}

	Verdi is a framework for practically verifying distributed systems. Often implementations of distributed systems are
	too complex to be exhaustively tested, so, Verdi attempts to choose an appropriate fault model to more effectively enumerate bugs and faults.

	%% Teaching Rigorous DS W/ Efficient Model Checking
	\item
	\textit{Teaching Rigorous Distributed Systems with Efficient Model Checking} \cite{MichaelWAET2019}

	While exhaustively determining bugs in a distributed system can be incredibly effective, it can, at the same time, be incredibly costly for developers.
	This paper purposes a model that allows students, or developers with fewer resources at their disposal to efficiently verify their systems and visually debug them.
	Also included, are methods to reduce the search space for potential faults in the system and to detect errors in realtime.

	%% Howard's Generalized Consensus Solution
	\item
	\textit{A Generalised Solution to Distributed Consensus} \cite{HowardGeneralized}

	This paper attempts to simplify the general consensus problem. It looks at the general consensus problem, and considers how it may be simplified in universal terms with respect to immutable state. They look specifically at the Paxos algorithm as an example. 
	It is synomomous with consensus, though, can be incredibly difficult to understand. This generalized solution to consensus hopes to quelle some of this confusion.
	In analysis, they find that quorum requirements of many algorithms could in fact be weakend.

\end{itemize}

\section{Consensus}

\begin{itemize}
	%% SDPaxos
	\item
	\textit{SDPaxos: Building Efficient Semi-Decentralized Geo-replicated State Machines} \cite{zhao2018sdpaxos}

	The distributed systems attemping geo-replication have run into multiple notorious problems: mainly load imbalance.
	SDPaxos proposes an alternative algorithm that is based on Paxos, that separates consensus into two distinct phases, replicating the commands to the nodes, and enforcing a consistent order on the nodes.
	This is done in an attempt to curb workload imbalance by maintaining optimal one-trip latency in two steps.

	%% Mencius
	\item{Mencius: building efficient replicated state machines for WANs \cite{Mencius}}

	%% Fast Paxos
	\item{Fast Paxos \cite{lamport2006fast}}

	%% Lamport's Generalized Consensus
	\item{Generalized Consensus and Paxos \cite{lamport2005generalized}}

	%% MDCC
	\item{MDCC: Multi-Data Center Consistency \cite{MDCC}}

	%% Flaw in EPaxos
	\item{On the correctness of Egalitarian Paxos \cite{SutraEPaxos}}

	%% EPaxos
	\item{There Is More Consensus in Egalitarian Paxos \cite{EPaxos}}

	%% FuzzyLog
	\item{The FuzzyLog: A Partially Ordered Shared Log \cite{FuzzyLog}}
\end{itemize}

\section{Databases and Implementations}

\begin{itemize}
	%% Spanner
	\item{Spanner: Google's Globally-Distributed Database \cite{Spanner}}

	%% Calvin
	\item{Calvin: fast distributed transactions for partitioned database systems \cite{Calvin}}
\end{itemize}

\bibliographystyle{acm}
\bibliography{refs}

\end{document}